\setcounter{chapter}{1}
\setcounter{section}{0}
\addcontentsline{toc}{chapter}{ĐẶT VẤN ĐỀ}

\chapter*{\begin{center}ĐẶT VẤN ĐỀ\end{center}}

\section{Lí do lựa chọn cơ sở đào tạo}
 Sau 28 năm xây dựng và trưởng thành, Đại học Thái Nguyên (ĐHTN) đã trở thành một đại học đa ngành, đa lĩnh vực, tạo dựng được uy tín lớn trong và ngoài nước. Phát huy tiềm lực của mình, trong những năm qua, ĐHTN đã đào tạo số lượng lớn cán bộ có trình độ thạc sĩ, tiến sĩ cho các tỉnh, các trường đại học, cao đẳng, trung học chuyên nghiệp và dạy nghề trên địa bàn vùng Trung du và miền núi phía Bắc. Số người tốt nghiệp thạc sĩ, tiến sĩ thuộc các địa phương vùng Trung du và miền núi phía Bắc chiếm trên 70,0\% số người học tốt nghiệp của ĐHTN. Hiện tại, ĐHTN đang triển khai đào tạo tại các cơ sở giáo dục đại học thành viên 115 ngành đào tạo sau đại học, bao gồm: 32 ngành tiến sĩ, 59 ngành thạc sĩ thuộc nhiều lĩnh vực khoa học. Chính vì vậy, có thể nói ĐHTN là một địa chỉ tin cậy cho các ứng viên muốn được tham gia học tập để đạt được học vị thạc sĩ, tiến sĩ.

Trường ĐH Công nghệ thông tin và truyền thông là đơn vị đào tạo thành viên thuộc ĐHTN. Trong 21 năm hình thành và phát triển, Trường luôn thực hiện mô hình phối hợp quản lý và đào tạo với Viện Công nghệ thông tin thuộc Viện Khoa học và Công nghệ Việt Nam để đào tạo nguồn nhân lực, nghiên cứu khoa học và chuyển giao công nghệ trong các lĩnh vực Công nghệ thông tin và truyền thông phục vụ sự nghiệp CNH, HĐH các tỉnh trung du và miền núi phía Bắc nói riêng, cả nước nói chung. Năm 2014, Trường được ĐHTN giao nhiệm vụ đào tạo trình độ tiến sĩ chuyên ngành Khoa học máy tính.

Từng là sinh viên của Trường Đại học Công nghệ thông tin và truyền thông, tốt nghiệp năm 2013, tôi đã có những ấn tượng vô cùng tốt đẹp về nhà trường. Đó là sự nhiệt tình, chuyên môn cao của đội ngũ giảng viên trong trường. Đó còn là cơ sở vật chất khang trang, hiện đại. Cũng vì thể cho nên sinh viên sau khi ra trường có khả năng tìm việc làm rất cao. Sau khi ra trường, tôi đã luôn mong muốn quay trở lại đây học tập nâng cao trình độ khi có điều kiện.

Năm 2020, tôi bắt đầu công tác tại Trung tâm Đào tạo từ xa - ĐHTN. Đây là một đơn vị thành viên của ĐHTN được thành lập theo quyết định số 466/QĐ-ĐHTN ngày 12/05/2012. Trong quá trình công tác tại đây, tôi cũng luôn nhận được sự động viên, khích lệ của các Thầy, Cô thuộc Trung tâm trong việc học tập và nâng cao trình độ. Vì vậy mà cùng năm 2020, tôi đã tham gia khóa học Thạc sỹ Khoa học máy tính tại Trường Đại học Công nghệ thông tin và truyền thông. Những ấn tượng tốt đẹp về nhà trường thời sinh viên của tôi lại càng thêm đậm nét trong quá trình học tập vừa qua. 

Trên cơ sở vừa đảm bảo nhiệm vụ học tập nâng cao trình độ, vừa đảm bảo thực hiện các nhiệm vụ do Trung tâm Đào tạo từ xa - ĐHTN giao phó, kết hợp với các điều kiện thuận lợi trong việc học tập sau đại học tại ĐHTN nói chung và Trường ĐH Công nghệ thông tin và truyền thông nói riêng, tôi có mong muốn được thực hiện công việc nghiên cứu sinh tại Trường ĐH Công nghệ thông tin và truyền thông. 

\section{Lí do lựa chọn lĩnh vực nghiên cứu}
Hiện nay, với sự phát triển của ngành CNTT và những định hướng phát triển của đất nước ta đối với ngành CNTT, ngành Khoa học máy tính đóng vai trò rất quan trọng trong việc xây dựng các nền tảng nghiên cứu, phát triển các công nghệ trong tương lai, đặc biệt là những nghiên cứu ứng dụng thiết thực trong cuộc sống.